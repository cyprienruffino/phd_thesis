\chapter{Domain-transfer with with auxiliary tasks for generative modeling}
\label{chap:chapter3}

\graphicspath{{images/chapter3/}, {tikz/chapter3/} }

\begin{chapterabstract}
	In this chapter, we tackle the problem of constrained image domain-transfer with generative models. We focus on the generation of images using Cycle-Consistent Generative Adversarial Networks (CycleGAN) with image domain constraints for converting \ac{RGB} images to polarimetry-encoded ones with constraints derived from the physics of polarimetry. Our work is driven by an application in road-scene object detection in polarimetric images. This is motivated by the application of deep learning frameworks to polarimetric imaging in various domains, including medical imaging and scene analysis.
	However, even if polarimetric imaging has shown improved performances on diverse tasks, such as object detection in road scenes images, their use may be hindered by reduced number of labeled training images. This issue could be resolved by data augmentation. Moreover, polarization modality is subject to some physical feasibility constraints that could be impeded standard classical data augmentation techniques. 
	Hence we propose a polarimetric image generation framework based on the CycleGAN approach to transfer \ac{RGB} images to polarimetry-encoded ones, in order to convert full labeled datasets to the polarimetric domain. We derive constraints from the optics of polarimetry that characterize the physical admissibility of a polarimetric image. By integrating these constraints as an auxiliary task at training stage, our \ac{GAN} learns to generate high-quality polarimetric images that follow the physics of polarimetry. This allows for transferring existing labeled \ac{RGB} datasets to the polarimetric domain without re-labeling of the data. 
	We evaluate the proposed generative model on road scene images. The obtained results achieved an effective generation of physical polarization-encoded images of high visual quality. The generated images are indeed coherent from a physics perspective. Further experiments on road object detection show that by training a detection model using a polarimetric images dataset that includes generated polarimetric images, the detection of cars and pedestrian are improved.
\end{chapterabstract}\\

\clearpage

The work in this chapter has led to the preparation of the following paper: 
\begin{itemize}
	\item Rachel Blin, Cyprien Ruffino, Samia Ainouz,  Gilles Gasso, Romain H\'erault, St\'ephane Canu and Fabrice Meriaudeau (June. 2020). Generating Polarimetric-encoded Images using Constrained Cycle-Consistent Generative Adversarial Networks.
	To be submitted.
\end{itemize}

\setcounter{minitocdepth}{3}
\minitoc
\setcounter{minitocdepth}{2}

%===========================================================
\section{Introduction}


In the previous chapters, we have seen that Generative  adversarial networks \citep{Goodfellow2014} are powerful deep generative models, able to learn complex data distributions and generate realistic samples from them. Arguably most of the impressive achievements of the \ac{GAN} were obtained for \ac{RGB} images but some works attempted to extend \ac{GAN} approaches to other less common imaging domains. Among these works, the task of generating images from the \ac{RGB} domain to these other imaging domains, using domain-translation approaches such as \ac{CycleGAN} \citep{Zhu2017a}. For instance, methods to generate infrared road scenes from \ac{RGB} counterpart images \citep{Zhang2018b}, to produce thermal images for person re-identification \citep{Kniaz2018} or for infrared image colorization \citep{Mehri2019}. In the same vein, \citet{Nie2017} achieved data augmentation in the field of medical imaging by transforming MRI inputs into pseudo-CT images and \citet{Sallab2019} used it to produce realistic \ac{LiDAR} points cloud from simulated ones. 

Following the previous stream of work, this chapter explores domain-transfer generative models on non-conventional imaging techniques. Specifically we investigate a  generative model framework to produce realistic polarimetric images from \ac{RGB} images.  The significant interest resides in the fact that polarimetric imaging is a rich modality that enables to characterize an object by its reflective properties. Those properties are object specific, hence, they convey strong features to analyze the content of a scene. In a polarimetric image, each pixel encodes information regarding the object's roughness, its orientation and its reflection \citep{Wolff1995}. Applications of polarimetric imaging range from indoor autonomous navigation \citep{Berger2017}, depth map estimation \citep{Zhu2019}, 3D object reconstruction \citep{Morel2006},  or early-stage cancer detection \citep{Rehbinder2016}. Also, polarization imaging was recently exploited in autonomous driving applications either to enhance car detection \citep{Fan2018}, road mapping and perception \citep{Aycock2017}, or to detect road objects in adverse weather conditions \citep{Blin2019}.  However, these  applications are characterized by the reduced size of the available training databases which restrains them from using deep neural networks, thus the need of polarimetric data generation model. 

Contrary to \ac{RGB}, \ac{LiDAR}, thermal or infrared image generation which mostly responded to visual qualitative  constraints, sampling polarization images is more challenging. Indeed, this imaging technique comes with physical admissibility constraints on the pixels of an image. As such, each pixel entry of such an image should satisfy some physical constraints related to light polarization principle and to the calibration setup of the acquisition devices.

Therefore, we formulate our problem of polarimetric image generation as a CycleGAN learning problem under physical constraints to ensure that the generated images are valid.  We study this problem in a fully unsupervised context, meaning that we do not have access to datasets of paired or labeled samples.  Techniques b based on cycle-consistency \citep{Zhu2017a} enabled to achieve unpaired image-to-image translation with a relatively few number of images. They allow to circumvent the expensive labeling issue in deep learning by transferring a source labeled dataset to one or multiple target domain \citep{Almahairi2018} by keeping unchanged the shapes of the source image. Starting from unpaired sets of RGB and polarimetric images, our proposed framework based on \ac{CycleGAN} \citep{Zhu2017a} is able to handle the physical polarization constraints during training. We demonstrate the effectiveness of our constrained-output CycleGAN on the KITTI\footnote{Karlsruhe Institute of Technology and Toyota Technological Institute}\citep{Geiger2012} and BDD100K datasets\footnote{the Berkeley Deep Drive dataset} \citep{Yu2020}, two common datasets used for object detection in road scenes. Using the generated polarization-encoded images to train a deep object detector, we witness an improvement of the detection performances of cars and pedestrians which are of great interest for autonomous driving applications. 

To summarize, the contributions of this chapter are:
\vspace{-10px}
\begin{itemize}
	\itemsep0em
	\item as far as our knowledge can go, we propose the first framework for generating physical polarization-encoded images starting from RGB images, 
	\item we propose a \ac{CycleGAN}-based model which allows to generate polarimetric-encoded images while handling the physical constraints the pixels of the generated image should satisfy,
	\item when plugged into the training procedure of an object detector for pretraining, the generated images help improving the detection performances.
\end{itemize}

The remainder of the chapter is organized as follows:  the polarization formalism and the physical constraints it involves are first presented in Section \ref{sec3:physical_prop}. Then, in Section \ref{sec3:related_works}, the formulation of the image-to-image translation from \ac{RGB} images to the polarimetric domain is described, and we review different approaches to tackle this problem, as well as their limitations. In Section \ref{sec3:solutions} a way to take into account these physical constraints during the training process of the CycleGAN for generating polarimetric images is investigated. Experimental evaluations are conducted in Section \ref{sec3:experiments}, in which we aim to translate RGB images of KITTI and BDD100K datasets into polarimetric images. We evaluate our approach as a data augmentation technique using an object detection network trained on the generated images. The last section concludes the chapter.

\CR{PERSPECTIVES ET PROXIMAL}

\section{Polarimetric imaging: formalism and constraints}
\label{sec3:physical_prop}

As most of this chapter revolves around polarimetric image generation, we first introduce the formalism of polarization that stems from the physics of polarimetry.  Polarization is a property of light that represents the direction of propagation of the electrical field of the light wave. Polarimetric imaging defines the polarization state of light waves reflected by each part of the scene. When an un-polarized light wave is being reflected, it becomes partially linearly polarized and its polarization depends on the normal surface  and the refractive index of the material it impinges on. As such, it is a different modality than classical color images, since it does not represent the wavelength of light, but contains rich information about the surfaces that the light reflected on, most notably information about the materials of these surfaces \citep{Gross2012}. In this section, we first propose an overview of the mathematical formulation of polarimetric imaging and then review the different physical constraints that apply to this imaging paradigm.

\subsection{Polarimetry-encoded images and Stokes vectors as parameters for polarization}

Similarly  to color images, several encoding formats exist for polarimetric images. The acquisition principle of a polarimetric camera is based on a set of polarizers located between the object and the sensors \citep{Bass1995}. In this work, we rely on a polarimetric image encoding format that consists in four channel images respectively obtained with four different linear polarizers oriented at $\alpha_\theta,  \theta\in\{1,...,4\} =$ (0\degree, 45\degree, 90\degree, 135\degree). The polarimetric camera captures an image $\vy \in \spaceY \subset \spaceR^{n \times p \times 4}$ consisting in the light intensities $\vy_{i,j_{\alpha_\theta}}$ of the scene for each angle $\alpha_\theta$ for each pixel $\vy_{i,j} = \begin{bmatrix} \vy_{i,j_{0}}& \vy_{{i,j}_{45}} & \vy_{{i,j}_{90}} & \vy_{{i,j}_{135}}\end{bmatrix}^\top$, $ \forall i\leq n,  j\leq p$. An example of the four different intensities for the same scene is shown in Figure~ \ref{fig:polar_overview intensities}. 

\begin{figure}
	\centering
	\begin{subfigure}{0.25\textwidth}
		\centering
		\includegraphics[width=\linewidth]{2474_I0.png}
	\end{subfigure}%
	\begin{subfigure}{0.25\textwidth}
		\centering
		\includegraphics[width=\linewidth]{2474_I45.png}
	\end{subfigure}%
	\begin{subfigure}{0.25\textwidth}
		\centering
		\includegraphics[width=\linewidth]{2474_I90.png}
	\end{subfigure}%
	\begin{subfigure}{0.25\textwidth}
		\centering
		\includegraphics[width=\linewidth]{2474_I135.png}
	\end{subfigure}
	\caption[Example of a polarimetric image]{Example of a polarimetric image. From left to right, the intensities corresponding to the polarizer rotation angles 0$\degree$, 45$\degree$, 90$\degree$ and 135$\degree$.}
	\label{fig:polar_overview intensities}
\end{figure}

The linearly-polarized reflected light can be described by measurable parameters, specifically by the linear Stokes vectors. These parameters are encoded as an image $\vs \in \spaceS \subset \spaceR^{n\times p\times 3}$ such that each pixel $\vs_{i,j}$ is a Stokes vector $\vs_{i,j} = \begin{bmatrix} \vs_{{i,j}_0} & \vs_{{i,j}_1} & \vs_{{i,j}_2} \end{bmatrix}^\top \in \spaceR^3$, $1 \leq i\leq n, 1\leq j\leq p$  . Here, $\vs_0>0$ represents the total light intensity, $\vs_1$ the amount of horizontally and vertically linearly polarized light and $\vs_2$ the amount of linearly polarized light at $\pm$~45\degree. 

Associated with each polarimetry encoding format is its so-called calibration matrix $\ma$ that allows for computing the Stokes vectors. In this work, the calibration matrix is set by the manufacturer of the polarimetric camera we use (a Polarcam\texttrademark 4D Technology\footnote{\href{https://www.4dtechnology.com}{https://www.4dtechnology.com}}) as
%
\begin{equation}
\ma = \frac{1}{2} {\begin{bmatrix}
		1 & \cos(2\alpha_1) & \sin(2\alpha_1) \\
		1 & \cos(2\alpha_2) & \sin(2\alpha_2) \\
		1 & \cos(2\alpha_3) & \sin(2\alpha_3) \\
		1 & \cos(2\alpha_4) & \sin(2\alpha_4)
\end{bmatrix}}
\\
=  \frac{1}{2} {\begin{bmatrix}
		1 & 1 & 0 \\
		1 & 0 & 1 \\
		1 & -1 & 0 \\
		1 & 0 & -1
\end{bmatrix}} \enspace. \nonumber
\label{eq3:calibration_matrix}
\end{equation}

Using $\ma\in\spaceR^{4\times3}$, we define\footnote{To ease the notation for the rest of this chapter, we use the matrix product notation $\mm\vt$ between a tensor $\vt \in \spaceR^{n\times p \times a}$ and  a matrix $\mm\in\spaceR^{a\times b}$ as computing a tensor $\vt' \in \spaceR^{n\times p \times b}$ such that each of its elements $\vt'_{i,j} = \mm\vt_{i,j}, \enspace1\leq i \leq n, 1 \leq j \leq p$.} the relationship between the Stokes vectors $\vs \in \spaceR^{n\times p \times 3}$ and the light intensities $\vy \in \spaceR^{n \times p \times 4}$ reaching the camera as
%
\begin{equation}
	\vy = \ma\vs\enspace.
	\label{eqn:IAS}
\end{equation}

To compute the Stokes parameters from the measured intensities (equation \ref{eqn:IAS}), we require $\ma^\dagger = (\ma^\top \ma)^{-1} \ma^\top \in \mathbb{R}^{3\times 4}$ the pseudo-inverse (or Moore-Penrose inverse) of the matrix $\ma$. The relationship between $\vs$ and $\vy$ is defined for each pixel as
%
\begin{equation}
	\vs_{i,j} = \ma^\dagger \vy_{i,j} 
	\forall i\leq n, j\leq p \nonumber
	\label{eqn:stokes2} \enspace.
\end{equation}

In our work, the pseudo-inverse $\ma_\dagger$ of the calibration matrix $\ma$ is
%
\begin{equation}
	\ma_\dagger = 
\begin{bmatrix}
	1 & 0 & 1 & 0 \\
	1 & 0 & -1 & 0 \\
	0 & 1 & 0 & -1
\end{bmatrix}
\enspace,
\end{equation}
%
thus we have the relation
%
\begin{equation}
	\vs_{i,j} =  \ma^\dagger \vy_{i,j}  =
	\begin{bmatrix}
		1 & 0 & 1 & 0 \\
		1 & 0 & -1 & 0 \\
		0 & 1 & 0 & -1
	\end{bmatrix}
	\begin{bmatrix} 
		\vy_{{i,j}_0} \\
		\vy_{{i,j}_{45}} \\
		\vy_{{i,j}_{90}} \\
		\vy_{{i,j}_{135}}
	\end{bmatrix} 
	= 
	\begin{bmatrix} 
		\vy_{0_{i,j}} + \vy_{{i,j}_{90}} \\
		\vy_{0_{i,j}} - \vy_{{i,j}_{90}} \\
		\vy_{{i,j}_{45}} - \vy_{{i,j}_{135}} 
	\end{bmatrix}\\
	1\leq i\leq n, 1 \leq j\leq p \nonumber \enspace .
\end{equation}

\subsection{Physical constraints of polarimetry}

A polarimetry-encoded image $\vy$ is deemed valid if its Stokes vectors satisfy two main conditions: they must be physically admissible and they must be the result of an acquisition process that uses the right calibration. Since we are interested in generating new polarimetric images, they will have to comply with these essential constraints. 

The three components of the Stokes vectors represent respectively the total light intensity, the intensity of the vertically and horizontally polarized light, and the intensity of the diagonally polarized light. To be physically admissible, the total light intensity $\vs_0$ of the Stokes vectors $\vs$ should be at least superior to the sum of the intensities of the diagonally, vertically and horizontally polarized light. Thus, we have

\begin{equation}
	\vs_0 \geq \sqrt{\vs_1^2 + \vs_2^2} \enspace .
\end{equation}
%rely on the degree of polarization (\ac{DOP}) \citep{Ainouz2013}, defined as

%\begin{equation}
%	\ac{DOP} = \frac{\sqrt{\vs_1^2+\vs_2^2}}{\vs_0} \enspace.
%\end{equation}

%The $\ac{DOP}\in [0,1]$ refers to the amount of polarized light in a wave, where a \ac{DOP} of 1  represents a totally polarized light, a null value corresponds to un-polarized light,  and a \ac{DOP} between 0 and 1 for partially polarized light. 

Additionally, since $\vs_0$ represents the total light intensity, it cannot be 0. Thus, to be physically admissible, a Stokes vector has to meet the conditions
%
\begin{equation}
	\vs_0 > 0
	\quad \mbox{ and } \quad 
	\vs_0^2 \geqslant \vs_1^2 + \vs_2^2 \enspace.
	\label{eqn:stokes_constraint_S0}
\end{equation}

Then, an additional check has to be done to ensure that a polarimetric image $\vy$  has been obtained using a given calibration matrix $\ma$.  To do so, we evaluate if the image $\vy$ can be reconstructed from the Stokes vectors computed using the pseudo-inverse $\ma^\dagger$ of the calibration matrix. By using equations (\ref{eqn:IAS}) and (\ref{eqn:stokes2}), we can formulate the condition  
%
\begin{equation}
\label{eq:calibration_constraint}
\vy = \ma\ma^\dagger \vy \enspace.
\end{equation}

Note that in the case where $\ma$ is invertible, $\ma^\dagger=\ma^{-1}$ so $\ma\ma^\dagger=Id$, thus this constraint is always enforced. In general, this constraint is satisfied if and only if $\vy \in \ker(\ma\ma^\dagger-Id)$ In the specific case where the calibration matrix $\ma$ defined in Equation \ref{eq3:calibration_matrix} is used, the solution to this constraint is 
%
\begin{equation}
\Big\{\vy = \begin{bmatrix}\vy_0 & \vy_{45} &  \vy_{90} & \vy_{135}\end{bmatrix}^\top\Big| \vy_0 + \vy_{90} = \vy_{45} + \vy_{135} \Big\} \enspace.
\end{equation}

The proof of this result is found in appendix \ref{app:physical_prop}. We finally obtain a set of three polarimetric constraints $\mathcal{C}_1$, $\mathcal{C}_2$ and $\mathcal{C}_3$ formulated as 
%
\begin{eqnarray}
	\label{eq3:constraints}
	\mathcal{C}_1 &:&\vy = \ma\ma^\dagger\vy, \\
	\mathcal{C}_2 &:& \vs_0^2 \geqslant \vs_1^2 +\vs_2^2 \enspace, \nonumber\\
	\mathcal{C}_3 &:& \vs_0 > 0 \enspace. \nonumber
\end{eqnarray}

In this chapter, we consider images that satisfy this set of constraints to be physically admissible.

%%%%%%%%%%%%%%%%%%%%%%%%%%%%%%%%%%%
%%%%%%%%%%%%%%%%%%%%%%%%%%%%%%%%%%%%%%%%%%%%%%%%%%%%

\section{Unsupervised color to polarimetric image translation}
\label{sec3:related_works}

In this section, we propose a formulation of the polarimetric image generation as a constrained domain-transfer problem. We examine the limits  of the classical domain-transfer approaches and propose an overview of some recent methods that overcome these limits. 

\subsection{Polarimetric image generation as a constrained conditional image generation problem}

The problematic studied in this chapter is learning a model for generating physically realistic polarimetry-encoded images from color images, using neither paired data nor labeled data. A polarimetric image generated that way should remain semantically consistent with the input color image, i.e it should represent the same scene and objects but in a different modality. Thus, there are two important aspects to this problem. First, we aim to learn a generative model $\G_{XY}$  such that, for an \ac{RGB} image $\vx \in \spaceX$ issued from the distribution $\p{X}$, the generated images $\G_{XY}(\vx) = \Hat{\vy} \in \spaceY$ are issued from $\p{Y}$ the distribution of the real polarimetric images. Hence, the generated images $\Hat{\vy}$ and  their Stokes vectors $\Hat{\vs} = \ma^\dagger\Hat{\vy}$ must respect the constraints $\mathcal{C}_1$, $\mathcal{C}_2$ and $\mathcal{C}_3$ (see \ref{eq3:constraints}).

We can formulate this as
%
\begin{eqnarray}
	\label{eq:polar_constr_formulation}
	\max_{\G_{XY}}& L(\G_{XY}) = \mathop{\mathbb{E}}_{\vx\sim \p{X}} \Big[\log (\p{Y}(\G_{XY}(\vx))\Big] \enspace \\
	\text{s.c.} & \G_{XY}(\vx_i) = \ma\vs_i \enspace;\enspace \vs_{0_i}^2 \geq \vs_{1_i}^2 +\vs_{2_i}^2  \enspace\text{and}\enspace  \vs_{0_i}^2 > 0   \nonumber \\
	\text{with}  & \vs_{i} = \ma^\dagger(\G_{XY}(\vx_{i})) \quad \forall i \nonumber \enspace.
\end{eqnarray}

These constraints enforce the physical admissibility of the generated polarimetric images, however they do not guarantee that the objects pictured in the generated images $\G_{XY(\vx_i)}$ will be the same as in the original images $\vx_i$. This property is called \textbf{semantic consistency} between the input $\vx$ and the generated image $\Hat{\vy}$ and is essential to the task of domain-transfer. Indeed, a model that is not semantically consistent could generate realistic images that are completely different from the provided inputs.

One final requirement is that the model should be trainable in an unpaired and unsupervised way. This implies that the only available datasets consist in unpaired and unlabeled samples $\setX = \{\vx_1, ..., \vx_s\}, \vx_i \in \spaceX$ and $\setY = \{\vy_1, ..., \vy_s\}, \vy_i \in \spaceY$ from the two domains $\spaceX $ and $\spaceY$. 

These two requirements can be approached using unsupervised domain-transfer methods.

\subsection{Approaches for unsupervised conditional domain-transfer}

In Section \ref{subs:domain_transfer}, we reviewed different approaches for unsupervised domain-transfer. Most notably, we introduced the cycle-consistency losses used in models such as \ac{CycleGAN} \citep{Zhu2017a}. These approaches consists in training two conditional \ac{GAN} models, $\G_{XY}: \spaceX \rightarrow \spaceY$ and $\G_{YX}: \spaceY \rightarrow \spaceX$, that maps samples between the distributions$\p{X}$ and $\p{Y}$ of the two domains, then training them with both the classical \ac{GAN} losses and the cycle-consistency loss, formulated as 
%
$$\esp{\vx\sim\p{\vx}} ||\vx - \Gyx(\Gxy(\vx))||_1 + \esp{y\sim\p{\vy}} ||\vy - \Gxy(\Gyx(\vy))||_1 \enspace .$$
%
The full \ac{CycleGAN}  problem can be summed up as 
%
\begin{align}
	\min_{\Gxy, \Gyx}\max_{\Dx, \Dy} & \lcycgan  (\G_{XY}, \G_{YX}, \D_X, \D_Y) = \nonumber \\ 
	\min_{\Gxy, \Gyx}\max_{\Dx, \Dy}  &\esp{\vx\sim\p{\vx}}   \Big[(1-\D_X(\vx)^2) + (\D_Y(\G_{XY}(\vx)))^2 \Big] + \esp{y\sim\p{\vy}}  \Big[(1-\D_Y(\vy))^2 + (\D_X(\Gyx(\vy)))^2 \Big] \nonumber \\
	& +\lambda \Big[ \esp{\vx\sim\p{\vx}} ||\vx - \Gyx(\Gxy(\vx))||_1 + \esp{y\sim\p{\vy}} ||\vy - \Gxy(\Gyx(\vy))||_1 \Big] \nonumber\enspace .
\end{align}
%
While these approaches allow for efficient domain-translation (and notably image-to-image translation), they do not integrate domain-specific knowledge, for example the polarimetric constraints mentioned in Section \ref{sec3:physical_prop}.

To constrain the domain-transfer process, several approaches rely on adding a task specific loss to the \ac{CycleGAN} objective. This can be done in a supervised way, leveraging on labeled data by training a task model and minimizing its error, or in an unsupervised way with an explicit task loss.

\textbf{TD-GAN} \citep{Zhang2018c} integrates a supervised conditioning process in a task of semantic segmentation from organ X-ray images. To do so, the authors rely on existing labeled datasets of digitally reconstructed radiographs (DRRs) and use them to train a Dense Image-to-Image (DI2I) \citep{Huang2018} semantic segmentation model. They use this model to condition a \ac{CycleGAN}-like model that translates images from the X-ray to the DRR domain by adding the segmentation loss to the \ac{CycleGAN} objective. In other words, DRR images are translated to X-ray, then back to DRRs, segmented using the pre-trained model and then compared to the ground truth segmentation using binary cross-entropy.  The full procedure is described in Figure \ref{fig:tdgan}.

\begin{figure}
	\centering
	\includegraphics[width=\textwidth]{tdgan}
	\caption{Overview of the TD-GAN approach. Figure from \citet{Zhang2018c}.}
	\label{fig:tdgan}
\end{figure}

\textbf{CyCADA} \citep{Hoffman2018} also implements the idea of using pre-trained classifiers or segmentation models to condition domain-transfer.  They achieve this conditioning by comparing the classes (or segmentation maps) of the source and generated images, and then adding the fitting term of these supervision models to the objective loss of a \ac{CycleGAN}-like model. While this conditioning process requires a model pre-trained in a supervised way, training the CyCADA model does not require labeled data. They demonstrate this approach for domain-transfer with datasets such as \ac{MNIST} \citep{LeCun1998a} and Street View House Numbers (SVHN) \citep{Netzer2011} for a digit classification task, and the SYNTHIA \citep{Ros2016}, GTA \citep{Richter2016} and Cityscapes \citep{Cordts2015} datasets for road scenes semantic segmentation The full method is illustrated in Figure \ref{fig:cycada}.


\begin{figure}
	\centering
	\includegraphics[width=\textwidth]{cycada}
	\caption[Cycle-consistent adversarial adaptation (CyCADA) overview]{Cycle-consistent adversarial adaptation overview. By directly remapping source training data into the target domain, they remove the low-level differences between the domains, ensuring that their task model is well-conditioned on target data. They depict here the image-level adaptation as composed of the pixel GAN loss ({\color{green}green}), the source cycle loss ({\color{red}red}), and the source and target semantic consistency losses (black dashed) – used when needed to prevent label flipping. For clarity the target cycle is omitted. The feature-level adaptation is depicted as the feature GAN loss ({\color{orange}orange}) and the source task loss ({\color{purple}purple}). Figure from \citet{Hoffman2018}.}
	\label{fig:cycada}
\end{figure}

Several methods also leverage on this conditioning approach to enhance the performance of the domain-transfer task. \textbf{VIGAN} \citep{Shang2017} includes a denoising auto-encoder; the aforementioned \textbf{CyCADA} an uses adversarial loss on the features extracted by a pre-trained classifier and \textbf{Attention-GAN} \citep{Chen2018b} leverages on an attention mechanism, both to increase the visual quality of the generated samples.

\section{Generating polarimetric images with auxiliary tasks for domain-transfer modeling}
\label{sec3:solutions}

The problem studied in this chapter is image-to-image translation from \ac{RGB} images to the polarimetry domain. In the previous sections, we formalized the constraints of polarimetric imaging and reviewed the different approaches for image-to-image translation based on generative modeling, as well as conditioning mechanism based on task-specific losses.

As  a main contribution in this chapter, we propose a \ac{CycleGAN}-based approach for conditioning the domain translation task with the constraints of polarimetry. We formulate a relaxation of both the calibration constraint (see Equation \ref{eq:calibration_constraint}) and the constraint of physical admissibility (see Equation \ref{eq:polar_constr_formulation}) and add the related costs to the \ac{CycleGAN} losses.

We evaluate this methods on a road-scene \ac{RGB} to polarimetric image translation. We compare the visual quality of the samples and the respect of the polarimetric constraints. We further examine the physical admissibility by studying the impact of using our generated images as a dataset for a road-scene detection task. We observe that our approach outperforms \ac{CycleGAN} on the used criterions and that using generated polarimetric images contributes to enhancing the performances on the road-scene detection task.

\subsection{Auxiliary tasks for color to polarimetric images domain-transfer}

As discussed above, to generate a realistic  polarimetric image  from an \ac{RGB} image, we propose to use the \ac{CycleGAN} approach to learn the translation models $\G_{XY}$ between $\spaceX$ the space of the polarimetric images and $\spaceY$ the RGB image domain. Let $\Hat{\vy} \in \spaceR^{n \times p \times 4}$ be a generated polarimetric image. To be physically admissible, it has to satisfy the admissibility constraints (\ref{eqn:stokes_constraint_S0}) and the calibration constraint (\ref{eq:calibration_constraint}). 

\begin{figure} 
	\centering
	\includegraphics[width=\textwidth*2/3]{ours_const}
	\caption{Overview of the CycleGAN training process extended with $L_{\mathcal{C}_1}$ and $L_{\mathcal{C}_2}$.}
	\label{fig:overview_polarCycle}
\end{figure}

By design, the first component of the Stokes vector is always positive as it represents the total intensity reflected from an object.  Indeed the last layer of the generation models customary uses the hyperbolic tangent as activation function, each output intensity $\Hat{\vy}$ is within the range $]-1,1[$ which we scale to $]0,255[$. Hence $\hat{\vs}_{0_{i,j}}=\hat{\vy}_{0_{i,j}}+\hat{\vy}_{{90_{i,j}}}$\\$, 1\leq~i~\leq~n,~1\leq~j~\leq~p$ (see equation (\ref{eqn:stokes2})) is ensured to be strictly positive pixel-wisely. Therefore, constraint $\mathcal{C}_3$ can be deemed satisfied for the real and the generated polarimetric images. To handle the remaining constraints $\mathcal{C}_1$ and $\mathcal{C}_2$, one could resort to the Lagrangian dual of \ac{CycleGAN} optimization problem (\ref{eq:cyclegan}) subject to these constraints. However, this may be computationally expensive, as it requires to entirely optimize four neural networks (respectively the discrimination and the mapping network models) in an inner loop of a dual ascent algorithm. Moreover the overall optimization procedure may not be stable because of the min-max game involved in the CycleGAN learning. In order to derive  an efficient algorithm to learn CycleGAN under output constraints, we introduce a relaxation of the problem. Instead of strictly enforcing the constraints, as in \citeq{eq:polar_constr_formulation}, we measure how far the generated image pixels are from the feasibility domain through additional cost functions we attempt to minimize.
%
For the constraint $\mathcal{C}_1$, a $\ell_2$ distance between the generated image $G_{YX}$ and $\ma\Hat{\vs}$ is proposed. It reads
%
\begin{equation}
L_{\mathcal{C}_1} (\G_{XY}) = \mathop{\mathbb{E}}_{\vx\sim \p{X}} ||\G_{XY}(\vx) - \ma\ma^\dagger\G_{XY}(\vx)||_2\enspace.
\label{eqn:ls}
\end{equation}
%
Similarly, to enforce the constraint $\mathcal{C}_2$, a rectified linear penalty $L_{\mathcal{C}_2}$ is considered. It is defined by
%
\begin{equation}
L_{\mathcal{C}_2} (\G_{XY}) = \mathop{\mathbb{E}}_{\vx\sim \p{X}}  \max\left(\Hat{\vs}_1^2 + \Hat{\vs}_2^2 -
\Hat{\vs}_0^2, 0 \right)\enspace,
\label{eqn:lreg}
\end{equation}
%
with $\hat{\vs} = \begin{bmatrix}	\Hat{\vs_0} & 	\Hat{\vs_1} & 	\Hat{\vs_2} \end{bmatrix}^\top = \ma^\dagger\G_{XY}(\vx)$.

The loss $L_{\mathcal{C}_1}$ translates the respect of the acquisition conditions according to the calibration matrix $A$ while  $L_{\mathcal{C}_2}$ is related to the physical admissibility constraint on the deduced Stokes vectors from the generated image. Gathering all these elements, we train our \ac{CycleGAN} under physical constraints, by optimizing the following objective function
%
\begin{equation}
	L_{final}(\G_{XY}, \G_{YX}, \D_X, \D_Y)= L_{CycleGAN}(\G_{XY}, \G_{YX}, \D_X, \D_Y)+\mu L_{\mathcal{C}_1}(\G_{XY}) + \nu L_{\mathcal{C}_2}(\G_{XY}) \enspace.
	\label{eqn:lfinal}
\end{equation}
%
\begin{algorithm}[!t]
	\begin{algorithmic}[]
		\REQUIRE{$\trainsetX$ and $\trainsetY$ two unpaired datasets, $\Gxy$ and $\Gyx$ the mapping networks, $\Dx$ and $\Dy$ the discrimination models, $b$ the mini-batch size, $\ma$ the calibration matrix and $\ma^\dagger$ its pseudo-inverse, $\lambda, \mu, \nu$ hyperparameters}
		\REPEAT
		\STATE sample a mini-batch $\lbrace \vx_i \rbrace_{i=1}^b$ from the RGB $\trainsetX$\;
		\STATE sample a mini-batch $\lbrace \vy_i \rbrace_{i=1}^b$ from the polarimetric $\trainsetY$\;
		\STATE update $\Dx$ by stochastic gradient descent of
		\STATE \ \ \ \ $ \sum_{i=1}^{m}\big(\Dx(\vx_i)-1)^2 + (\Dx(\Gyx(\vy_i))\big)^2$
		\STATE update $\Dy$ by stochastic gradient descent of
		\STATE \ \ \ \ $ \sum_{i=1}^{m}\big(\Dy(\vy_i)-1)^2 + (\Dy(\Gxy(\vx_i))\big)^2$
		\STATE \textbf{for} $i=1$ to $b$, compute $\hat{\vs_i} = \begin{bmatrix}	\Hat{\vs_{0_i}} & 	\Hat{\vs_{1_i}} & 	\Hat{\vs_{2_i}}\end{bmatrix}^\top = \ma^\dagger\G_{XY}(\vx_i)$.
		\STATE update $\Gxy$ by stochastic gradient descent of
		\STATE \ \ \ \ $ \sum_{i=1}^n\big(\Dy(\Gxy(\vx_i))-1\big)^2 + \lambda \big(||\vx_i - \Gyx(\Gxy(\vx_i)\big)||_1 +||\vy_i -\Gxy(\Gyx(\vy_i))||_1\big)$
		\STATE \ \ \ \ \ \ $+ \mu \big(\|\vx_i - \ma\ma^\dagger\G_{XY}(\vx_i)\|^2_F\Big)+ \nu \big(\max(\vs_{1_i}^2 + \vs_{2_i}^2 - \vs_{0_i}^2, 0\big)$
		\STATE update $\Gyx$ by stochastic gradient descent of
		\STATE \ \ \ \ $ \sum_{i=1}^n \big(\Dx(\Gyx(\vy_i))-1\big)^2+ \lambda \big(||\vx_i - \Gyx(\Gxy(\vx_i))||_1 + ||\vy_i - \Gxy(\Gyx(\vy_i))||_1\big)$\;
		\UNTIL a stopping condition is met
	\end{algorithmic}
	\caption{CycleGAN with relaxed constraints training algorithm}
	\label{alg:cyclegan_relaxed_train}
\end{algorithm}
%
The non-negative hyper-parameters $\mu$ and $\nu \in \mathbb{R}^{+}$ control respectively the balance of admissibility and calibration constraints according to the CycleGAN loss $L_{CycleGAN}$ (see equation~\eqref{eq:cyclegan}). As the values of $L_{\mathcal{C}_1}$ and $L_{\mathcal{C}_2}$ are computed pixel-wisely, we consider their averages over the whole images in the objective function. The training principle of the proposed generative model is illustrated in Figure~\ref{fig:overview_polarCycle} and detailed in Algorithm~\ref{alg:cyclegan_relaxed_train}.

\subsection{Experimental evaluation}
\label{sec3:experiments}

Hereafter, the experimental setup, including the image generation procedure and its evaluation, is presented. 

\subsubsection{-  Experimental setup} \label{subsec:polar_gen}

To conduct the experiments, we rely on the polarimetric dataset presented in \citep{Blin2020} whose details are summarized in Table \ref{tab:dataset_properties}. From this dataset we select 2485 unpaired images from each domain (RGB and polarimetry). Example instances are shown in Figures~\ref{fig:polar_example} and~\ref{fig:rgb_example}  for polarimetric and RGB images respectively. The polarimetric images are of dimension $500 \times 500 \times 4$. The latter dimension is due to the four intensities acquired by the camera, namely $\vy_0, \vy_{45}, \vy_{90}$ and $\vy_{135}$. The RGB images are of dimension $906 \times 945 \times 3$.
\begin{figure}[t]
	\centering
	\begin{subfigure}{.2\textwidth}
		\centering
		\includegraphics[width=\linewidth]{1625_I0.png}
	\end{subfigure}%
	\begin{subfigure}{.2\textwidth}
		\centering
		\includegraphics[width=\linewidth]{240_I0.png}
	\end{subfigure}%
	\begin{subfigure}{.2\textwidth}
		\centering
		\includegraphics[width=\linewidth]{2474_I0.png}
	\end{subfigure}%
	\begin{subfigure}{.2\textwidth}
		\centering
		\includegraphics[width=\linewidth]{48_I0.png}
	\end{subfigure}%
	\begin{subfigure}{.2\textwidth}
		\centering
		\includegraphics[width=\linewidth]{766_I0.png}
	\end{subfigure}
	\caption[Examples of images in the polarimetric dataset ]{Examples of images in the polarimetric dataset \citep{Blin2020}. Only the intensities $\vy_0$ are shown here.}
	\label{fig:polar_example}
\end{figure}
\begin{figure}
	\centering
	\begin{subfigure}{.2\textwidth}
		\centering
		\includegraphics[width=\linewidth]{0030144.png}
	\end{subfigure}%
	\begin{subfigure}{.2\textwidth}
		\centering
		\includegraphics[width=\linewidth]{0038544.png}
	\end{subfigure}%
	\begin{subfigure}{.2\textwidth}
		\centering
		\includegraphics[width=\linewidth]{0025879.png}
	\end{subfigure}%
	\begin{subfigure}{.2\textwidth}
		\centering
		\includegraphics[width=\linewidth]{0032059.png}
	\end{subfigure}%
	\begin{subfigure}{.2\textwidth}
		\centering
		\includegraphics[width=\linewidth]{0088999.png}
	\end{subfigure}
	\caption{Examples of images in the RGB dataset.}
	\label{fig:rgb_example}
\end{figure}

\begin{table}
	\begin{center}
		\begin{tabular}{c c c c c}
			\Bigrule
			& Train & Val & Test \\
			\bigrule
			Images & 3861 & 1248 & 509 \\
			\bigrule
			car & 19587 & 3793 & 2793 \\
			person & 2049 & 294 & 161 \\
			bike & 16 & 35 & 3 \\
			motorbike & 52 & 4 & 5 \\
		\end{tabular}
		\caption[Polarimetric dataset features]{Polarimetric dataset features. The bottom rows indicate the total number of instances within each class.}
		\label{tab:dataset_properties}
	\end{center}
\end{table}

Our \ac{CycleGAN}s were trained for 400 epochs on randomly cropped patches of size $200\times 200$, as recommended for CycleGAN \citep{Zhu2017a}. As for the constraints, we found experimentally that setting the hyper-parameters $\mu = 1$ and $\nu = 1$ in equation \eqref{eqn:lfinal} provides the best performances. As for the original CycleGAN, the hyper-parameter $\lambda$, controlling the reconstruction cost, was set to $\lambda = 10$. The learning rate is decreased linearly from $2 \times 10^{-4}$ to $2 \times 10^{-6}$ during the epochs.

To evaluate the effectiveness of the generative model, we consider KITTI \citep{Geiger2012} and BDD100K \citep{Yu2020}(only using daytime images since polarimetry fails to characterize objects during nighttime) which often serve as test-bed in applications related to road scene object detection. The constrained-output CycleGANs we train are used to transfer RGB images from KITTI and BDD100K to the polarimetric domain. The resulting datasets are denoted respectively as Polar-KITTI and Polar-BDD100K. Since the CycleGAN architecture is fully convolutional, it has no requirement on the size of the input image. Therefore, even if the model was trained on $200 \times 200$ patches, it scales straightforwardly to the images of size $1250 \times 375$ from KITTI and of size $1280 \times 720$ from BDD100K datasets.

To assess whether or not fulfilling the physical  constraints is paramount, we investigate a variant of Polar-KITTI and Polar-BDD100K: we learn a standard unconstrained CycleGAN based on the same unpaired RGB/polarimetric images. It is worth mentioning that the so generated polarization-encoded images do not mandatory satisfy the feasibility constraints. 

\subsubsection{- Evaluation of the generated images} \label{subsec:eval_gen_img}

In order to assert the ability of the generated Polar-KITTI and Polar-BDD100K datasets to preserve the relevant features for road scene applications, we train a detection network following the setup in Figure~\ref{fig:experimental_setup}. For this experiment, a RetinaNet-50 \citep{Lin2017} pre-trained on the MS COCO dataset \citep{Lin2014} is fine-tuned in two different settings. In the first setup the detection model is fine-tuned based on the original RGB KITTI (or BDD100K) while the second experimental setting considers the fine-tuning on the generated polarimetric images from KITTI (Polar-KITTI) or BDD100K (Polar-BDD100K) datasets. Afterwards the final detection models are obtained in both settings by a final fine-tuning on the real polarimetric dataset (see Table \ref{tab:dataset_properties}). The same experiments were carried out for the unconstrained variant of the generated images.

\begin{figure}
	\centering
	\vspace{-1cm}
	\includegraphics[width=\textwidth*4/5]{Final_experiment_correction_1.png}\\
	\hrulefill\vspace{15pt}\par
	\includegraphics[width=\textwidth]{Final_experiment_correction_2.png}\\
	\hrulefill\vspace{15pt}\par
	\includegraphics[width=\textwidth]{Final_experiment_correction_3.png}
	\caption[Setup of the detection evaluation experiment]{Setup of the detection evaluation experiment. The procedure is illustrated with the KITTI dataset and straightforwardly extends to the BDD100K dataset. Top: domain-transfer procedure with our model; Center: baseline setup; Bottom: our setup}
	\label{fig:experimental_setup}
\end{figure}

Overall, the trained \ac{CycleGAN}s and detection networks under these settings are evaluated in qualitative and quantitative ways. The end goal is to check the ability of the generated images to help learning polarimetry-based features for object detection, and the influence of respecting the polarimetric feasibility constraints on detection performances.

We  measure the visual quality of the generated images by computing the classical Fréchet Inception Distance \citep{Heusel2017} (see Section \ref{subs:evaluation_methods}). Computing this distance requires to extract visual features from each set of images (real and generated) using a pre-trained deep neural network (usually an Inception v3 \citep{Szegedy2016} network pre-trained on \ac{ImageNet} \citep{Deng2009}) and to evaluate the Fréchet (or Wasserstein) distance between the distributions of these features, which are assumed Gaussian distributions (thoroughly explained in \citesec{subs:evaluation_methods}). We compute this distance using 500 images from each generated polarimetric dataset and from the test set as described in Table \ref{tab:dataset_properties}.

As feature extractor, since the classical Inception v3 network is not adapted to polarimetric images, we use the convolutional part of a polarimetry-adapted RetinaNet detection network \citep{Blin2019}, which has been trained on the MS-COCO dataset and fine-tuned on a real polarimetric dataset.
%
In order to evaluate the improvements in the detection, we compute the error rate evolution $ER_o$. The improvement $ER_o$ on the detection of the object $o$ is given by:
$$
ER_o = \frac{\Big(1 - \ac{AP}_o^{p}\Big) -\Big (1 - AP_o^{RGB}\Big)}{1-AP_o^{RGB}}\enspace,
$$

\noindent where $\ac{AP}_o^{RGB}$ and $AP_o^{p}$ respectively denote the average precision for object $o$ detection in \ac{RGB} and in polarimetric images.

\subsubsection{- Results and discussion}
\label{subs3:results}

First we evaluate whether the generated images are qualitatively coherent or not. For the sake, we reconstruct the polarimetric images from their RGB generation, which refers to $\G_{XY} \circ \G_{YX}$. The reconstruction of these RGB images is shown in Figure~\ref{fig:reco_polar}. 
% A visual comparison of the same generated polarimetric image with and without constraints is illustrated in Figure~\ref{fig:generated_kitti}. As can be seen, the scene content of the generated images is preserved.
%\RB{La Figure 7 est-elle vraiment pertinente en fin de compte? Elle prend beaucoup de place et au final on ne comprend pas ce qu'elle cherche à démontrer}

\begin{figure}
	\centering
	% \includegraphics[width=\linewidth]{reco_polar.png}
	\begin{subfigure}{.11\textwidth}
		\centering
		\includegraphics[width=\linewidth]{0001611_I0.png}
	\end{subfigure}%
	\begin{subfigure}{.11\textwidth}
		\centering
		\includegraphics[width=\linewidth]{0001611_I45.png}
	\end{subfigure}%
	\begin{subfigure}{.11\textwidth}
		\centering
		\includegraphics[width=\linewidth]{0001611_I90.png}
	\end{subfigure}%
	\begin{subfigure}{.11\textwidth}
		\centering
		\includegraphics[width=\linewidth]{0001611_I0.png}
	\end{subfigure}%
	\begin{subfigure}{.105\textwidth}
		\centering
		\includegraphics[width=\linewidth]{0003024.png}
	\end{subfigure}%
	\begin{subfigure}{.105\textwidth}
		\centering
		\includegraphics[width=\linewidth]{I0_0003024.png}
	\end{subfigure}%
	\begin{subfigure}{.105\textwidth}
		\centering
		\includegraphics[width=\linewidth]{I45_0003024.png}
	\end{subfigure}%
	\begin{subfigure}{.105\textwidth}
		\centering
		\includegraphics[width=\linewidth]{I90_0003024.png}
	\end{subfigure}%
	\begin{subfigure}{.105\textwidth}
		\centering
		\includegraphics[width=\linewidth]{I135_0003024.png}
	\end{subfigure}
	\begin{subfigure}{.11\textwidth}
		\centering
		\includegraphics[width=\linewidth]{0026648_I0.png}
	\end{subfigure}%
	\begin{subfigure}{.11\textwidth}
		\centering
		\includegraphics[width=\linewidth]{0026648_I45.png}
	\end{subfigure}%
	\begin{subfigure}{.11\textwidth}
		\centering
		\includegraphics[width=\linewidth]{0026648_I90.png}
	\end{subfigure}%
	\begin{subfigure}{.11\textwidth}
		\centering
		\includegraphics[width=\linewidth]{0026648_I135.png}
	\end{subfigure}%
	\begin{subfigure}{.105\textwidth}
		\centering
		\includegraphics[width=\linewidth]{0033019.png}
	\end{subfigure}%
	\begin{subfigure}{.105\textwidth}
		\centering
		\includegraphics[width=\linewidth]{I0_0033019.png}
	\end{subfigure}%
	\begin{subfigure}{.105\textwidth}
		\centering
		\includegraphics[width=\linewidth]{I45_0033019.png}
	\end{subfigure}%
	\begin{subfigure}{.105\textwidth}
		\centering
		\includegraphics[width=\linewidth]{I90_0033019.png}
	\end{subfigure}%
	\begin{subfigure}{.105\textwidth}
		\centering
		\includegraphics[width=\linewidth]{I135_0033019.png}
	\end{subfigure}
	\begin{subfigure}{.11\textwidth}
		\centering
		\includegraphics[width=\linewidth]{0033248_I0.png}
	\end{subfigure}%
	\begin{subfigure}{.11\textwidth}
		\centering
		\includegraphics[width=\linewidth]{0033248_I45.png}
	\end{subfigure}%
	\begin{subfigure}{.11\textwidth}
		\centering
		\includegraphics[width=\linewidth]{0033248_I90.png}
	\end{subfigure}%
	\begin{subfigure}{.11\textwidth}
		\centering
		\includegraphics[width=\linewidth]{0033248_I135.png}
	\end{subfigure}%
	\begin{subfigure}{.105\textwidth}
		\centering
		\includegraphics[width=\linewidth]{0040999.png}
	\end{subfigure}%
	\begin{subfigure}{.105\textwidth}
		\centering
		\includegraphics[width=\linewidth]{I0_0040999.png}
	\end{subfigure}%
	\begin{subfigure}{.105\textwidth}
		\centering
		\includegraphics[width=\linewidth]{I45_0040999.png}
	\end{subfigure}%
	\begin{subfigure}{.105\textwidth}
		\centering
		\includegraphics[width=\linewidth]{I90_0040999.png}
	\end{subfigure}%
	\begin{subfigure}{.105\textwidth}
		\centering
		\includegraphics[width=\linewidth]{I135_0040999.png}
	\end{subfigure}
	\begin{subfigure}{.11\textwidth}
		\centering
		\includegraphics[width=\linewidth]{0048448_I0.png}
	\end{subfigure}%
	\begin{subfigure}{.11\textwidth}
		\centering
		\includegraphics[width=\linewidth]{0048448_I45.png}
	\end{subfigure}%
	\begin{subfigure}{.11\textwidth}
		\centering
		\includegraphics[width=\linewidth]{0048448_I90.png}
	\end{subfigure}%
	\begin{subfigure}{.11\textwidth}
		\centering
		\includegraphics[width=\linewidth]{0048448_I135.png}
	\end{subfigure}%
	\begin{subfigure}{.105\textwidth}
		\centering
		\includegraphics[width=\linewidth]{0059179.png}
	\end{subfigure}%
	\begin{subfigure}{.105\textwidth}
		\centering
		\includegraphics[width=\linewidth]{I0_0059179.png}
	\end{subfigure}%
	\begin{subfigure}{.105\textwidth}
		\centering
		\includegraphics[width=\linewidth]{I45_0059179.png}
	\end{subfigure}%
	\begin{subfigure}{.105\textwidth}
		\centering
		\includegraphics[width=\linewidth]{I90_0059179.png}
	\end{subfigure}%
	\begin{subfigure}{.105\textwidth}
		\centering
		\includegraphics[width=\linewidth]{I135_0059179.png}
	\end{subfigure}
	\caption[Examples  of polarimetric image reconstruction]{Examples  of polarimetric image reconstruction. From left to right: $\vy_0$, $\vy_{45}$, $\vy_{90}$ and $\vy_{135}$ ground truth, RGB image and $\vy_0$, $\vy_{45}$, $\vy_{90}$ and $\vy_{135}$ generated from RGB image, using the model trained with relaxed constraints.}
	\label{fig:reco_polar}
\end{figure}

% \begin{figure}
%     \centering
%     \includegraphics[width=\linewidth]{comparison_generation.png}
%     \caption{Example of generated images (Polar-KITTI). The left column represents the generation with constraints and the right column refers to image generation without constraints. From top to bottom: $I_0$, $I_{45}$, $I_{90}$ and $I_{135}$.}
%     \label{fig:generated_kitti}
% \end{figure}

As for the constraints, Table \ref{tab:polar_constraints} shows how including them to the \ac{CycleGAN}'s loss helps to generate  images which better fulfill the physical polarimetric properties at the pixel scale. The errors related to the constraints $\mathcal{C}_1$ and $\mathcal{C}_2$ on generated images using our approach are consistent with the observed errors on the real images (which corresponds to acquisition errors), whereas the unconstrained approach yields poor results. Obviously, constraint $\mathcal{C}_3$ is met for all generated images thanks to the $\tanh$ activation at the last layer of the generative models. Additionally, the obtained Fréchet Inception Distances (see table \ref{tab:polar_constraints}) indicates that taking the constraints into account improves visual quality and physical admissibility of the generated samples on the test set.


\begin{table}
	\begin{center}
		\begin{tabular}{c c c c || c}
			\Bigrule
			Datasets & $\mathcal{C}$ & Mean & Median & FID \\
			\Bigrule
			Real & $\mathcal{C}_1$ & 0.06 $\pm$ 0.04 & 0.04  \\
			polar & $\mathcal{C}_2$ & 2.47 $\pm$ 7.11\% & 0.48\% & N/A \\
			& $\mathcal{C}_3$ & 0\% & 0\% \\
			\bigrule 
			Generated & $\mathcal{C}_1$ & 0.26 $\pm$ 0.19 & 0.23 \\
			polar no $\mathcal{C}$ & $\mathcal{C}_2$ & 27.31 $\pm$ 43.5\% & 2.15\% & 6022.7\\
			& $\mathcal{C}_3$ & 0\% & 0\% \\
			\bigrule 
			Relaxed & $\mathcal{C}_1$ & 0.12 $\pm$ 0.04 & 0.12 \\
			constraints  & $\mathcal{C}_2$ & 1.55 $\pm$ 3.36\% & 0.14\% &\textbf{4485.1}\\
			& $\mathcal{C}_3$ & 0\% & 0\% \\
		\end{tabular}
		\caption[Evaluation of the generated images]{Evaluation of the constraint fulfillment using the designed losses $L_{\mathcal{C}_1}$ and $L_{\mathcal{C}_2}$ at the pixel scale, and  the visual quality using the Fréchet Inception Distance (\ac{FID}). Note that the scale of the \ac{FID} scores computed with the pre-trained RetinaNet is larger than when using a pre-trained Inception v3 network. Here, the column $\mathcal{C}$ indicates the evaluated constraint. $\mathcal{C}_1$ refers to the constraints $\vy = \ma\ma^\dagger\vy$, $\mathcal{C}_2$ to $\vs_0^2 \geqslant \vs_1^2 + \vs_2^2$ and $\mathcal{C}_3$ to $\vs_0 > 0$. The mean and the median of the percentage of pixels in an image that do not fulfill the constraints $\mathcal{C}_2$ and $\mathcal{C}_3$ are computed. Regarding the constraint $\mathcal{C}_1$, we compute the mean and the median of $||\vy - \ma\ma^\dagger\vy|| / (||\vy|| + ||\ma\ma^\dagger\vy||)$.}
		\label{tab:polar_constraints}
	\end{center}
\end{table}

Next, we show the benefit of the generated images in object detection task, enabling to verify that objects within them are globally physically coherent. 
The RetinaNet-based detection model were trained according to the setups previously described (see Figure \ref{fig:experimental_setup}) and the obtained detection performances in term of mean average precision ($\ac{mAP}$) are summarized in Table \ref{tab:obtained_results}. We choose not to evaluate the bike and motorbike detection performances as the polarimetric dataset does not contain enough objects of those two classes.

\begin{table}
	\begin{center}
		\begin{tabular}{c c c c| c c cc}
			\Bigrule
			Databases & Class & Test & $ER_o$ & Databases	 & Class & Test & $ER_o$ \\
			\Bigrule
			KITTI RGB & person & 0.663 & N/A & BDD100K RGB & person & 0.736 & N/A \\
			+ real polar & car & 0.785 & N/A & + real polar & car & \textbf{0.821} & N/A \\
			&$mAP$ & 0.724 & N/A & &$mAP$ & 0.778 & N/A \\
			\bigrule 
			Polar-KITTI  & person & 0.673 & -0.03 & Polar-BDD100K  & person & 0.720 & 0.06 \\
			no $\mathcal{C}$ + real polar & car & 0.786 & -0.01 & no $\mathcal{C}$ + real polar & car & 0.816 & 0.03 \\
			&$mAP$ & 0.730 & -0.02 & &$mAP$ & 0.768 & 0.05 \\
			\bigrule 
			Polar-KITTI with  & person & \textbf{0.704} & -0.12 & Polar-BDD100K  & person & \textbf{0.762} & -0.10 \\
			relaxed $\mathcal{C}$  & car & \textbf{0.794} & -0.04 & with relaxed $\mathcal{C}$ & car & 0.815 & 0.03 \\
			&$mAP$ & \textbf{0.749} & -0.09 & &$mAP$ & \textbf{0.789} & -0.05 \\
		\end{tabular}
	\end{center}
	\caption[Comparison of the detection performance after successive fine-tunings]{
	Comparison of the detection performance after the two successive fine-tunings. RetinaNet-50 pre-trained on MS COCO is the baseline of all experiments. The first row refers to the RetinaNet-50 fine-tuned on KITTI or BDD100K RGB. The second row refers to the fine-tuning on Polar-KITTI or Polar-BDD100K without physical constraints and the bottom row represents the detection model fine-tuned on Polar-KITTI or Polar-BDD100K with enforced constraints. Every model is finally fine-tuned on the real polarimetric dataset. 
	}
	\label{tab:obtained_results}
\end{table}

As we can see in Table \ref{tab:obtained_results}, using the generated images improves the detection performance in real polarimetric images. The improvement is substantial for car and pedestrian detection. We achieve an improvement of 4\% for car detection and of 12\% for pedestrian detection which leads to an overall improvement of 9\% in the detection, using Polar-KITTI with constraints. Similarly for Polar-BDD100K dataset, we notice an improvement of 10\% for pedestrian detection which leads to an increased $\ac{mAP}$ of 5\% (pedestrians and cars). However, we notice that for BDD100K similar detection performances are obtained either for RGB or polarimetric images and this is due to the fact that generated images using CycleGANs do not perform well on small objects. To verify that, we compare the evolution of the detection scores while setting a minimal area to the bounding boxes to be detected. The results of this experiment are shown for the training including the Polar-BDD100K and the RGB BDD100K in Figure~\ref{fig:bounding_boxes}.The results of this experiment illustrate that when the minimal area of bounding boxes increases the $\ac{AP}$ of car regarding the training including Polar-BDD100K overcomes the one including RGB BDD100K. We can thus conclude that the limit of this work is the low quality of the small objects in the generated images. 

\begin{figure}
	\centering
	\includegraphics[width=\linewidth]{area_bounding_boxes_study.png}
	\caption[Evolution of the average precision when setting a minimal area of the bounding boxes]{Evolution of the average precision when setting a minimal area of the bounding boxes to be detected. Here green lines refer to the evolution of cars' detection, blue lines to the evolution of the $\ac{mAP}$ and red lines to the evolution of person's detection. The dashed lines refer to the training including the BDD100K RGB and the solid lines to the training including Polar-BDD100K.}
	\label{fig:bounding_boxes}
\end{figure}

\section{Perspectives}
\label{sec:projector}

In this section, we propose to explore some perspectives and new approaches for transferring color images to the polarimetric domain. We formulate an operator for projecting generated images onto the space delimited by the constraints and propose two algorithms based on this projector. Since these approaches are, at the moment of writing this thesis, work-in-progress, no experimental evaluations are proposed for these approaches.

\subsection{Generating polarimetric images with a projector operator}

In the same fashion as in Section \ref{sec3:solutions}, we aim to generate images $ \Hat{\vy} =  \G_{XY}(\vx)$, where $\vx \in \spaceX$ is a sample from the \ac{RGB} domain, such that $\Hat{\vs} = \ma^\dagger\Hat{\vy} \in  \spaceS$ the space of the Stokes vectors. Each of the vectors must  respect $\mathcal{C}_1$, $\mathcal{C}_2$ and $\mathcal{C}_3$, which in fact correspond to a second-order cone, or Lorentz cone \citep{Boyd2004}. Thus, let
%
\begin{equation}
	\setC = \left \lbrace (\vs_0,\vs_{1,2}) \in  \spaceS \,\, \Big| \,\, \|\vs_{1,2}\|_2 \leq \vs_0, \,\, \vs_{1,2} = \begin{bmatrix} \vs_1 \\ \vs_2 \end{bmatrix} \right \rbrace \enspace,
\end{equation}
%	
\begin{algorithm}[!t]
	\begin{algorithmic}[]
		\REQUIRE{$\trainsetX$ and $\trainsetY$ two unpaired datasets, $\Gxy$ and $\Gyx$ the mapping networks, $\Dx$ and $\Dy$ the discrimination models, $m$ the mini-batch size, $\ma$ the calibration matrix and $\ma^\dagger$ its pseudo-inverse, $\lambda, \mu$ hyperparameters}
		\REPEAT
		\STATE sample a mini-batch $\lbrace \vx_i \rbrace_{i=1}^m$ from $\trainsetX$\;
		\STATE sample a mini-batch $\lbrace \vy_i \rbrace_{i=1}^m$ from $\trainsetY$\;
		\STATE update $\Dx$ by stochastic gradient descent of
		\STATE \ \ \ \ $ \sum_{i=1}^{m}\big(\Dx(\vx_i)-1)^2 + (\Dx(\Gyx(\vy_i))\big)^2$
		\STATE update $\Dy$ by stochastic gradient descent of
		\STATE \ \ \ \ $ \sum_{i=1}^{m}\big(\Dy(\vy_i)-1)^2 + (\Dy(\ma\Pi_{\setC}(\ma^\dagger\Gxy(\vx_i)))\big)^2$
		\STATE update $\Gxy$ by stochastic gradient descent of
		\STATE \ \ \ \ $ \sum_{i=1}^n\big(\Dy(\Gxy(\vx_i))-1\big)^2$ $+ \mu \big(\ma\ma^\dagger\G_{XY}(\vx_i)\Big)$
		\STATE \ \ \ \ \ \ $ + \lambda \big(||\vx_i - \Gyx(\ma\Pi_{\setC}(\ma^\dagger\Gxy(\vx_i))\big)||_1 +||\vy_i -\ma\Pi_{\setC}(\ma^\dagger\Gxy(\Gyx(\vy_i)))||_1\big)$
		\STATE \ \ \ \ \ \ $ + \mu \big(\|\vx_i - \ma\ma^\dagger\G_{XY}(\vx_i)\|^2_F\Big)$
		\STATE update $\Gyx$ by stochastic gradient descent of
		\STATE \ \ \ \ $ \sum_{i=1}^n \big(\Dx(\Gyx(\vy_i))-1\big)^2 $
		\STATE \ \ \ \ \ \ 	$+ \lambda \big(||\vx_i - \Gyx(\ma\Pi_{\setC}(\ma^\dagger\Gxy(\vx_i)))||_1 + ||\vy_i -\ma\Pi_{\setC}(\ma^\dagger\Gxy(\Gyx(\vy_i))||_1\big)$\;
		\UNTIL a stopping condition is met
	\end{algorithmic}
	\caption{Training algorithm for CycleGAN with projected images}
	\label{alg:cyclegan_train_projection}
\end{algorithm}

\begin{figure} 
	\centering
	\includegraphics[width=\textwidth]{ours_proj}
	\caption[Overview of the CycleGAN training process with extended the projection operator]{Overview of the CycleGAN training process with extended the projection operator. The $L_{\setC_1}$ term is omitted.}
	\label{fig:overview_polarCycle_proj}
\end{figure}

a convex set whose vectors satisfy the aforementioned constraints. As such, we can reformulate the Problem (\ref{eq:polar_constr_formulation}) as
%	
\begin{eqnarray}
	\label{eq:polar_set_formulation}
	\max_G& L(G) = \mathop{\mathbb{E}}_{\vx\sim \p{X}} \Big[\log (\p{Y}(\G_{XY}(\vx))\Big]  \\
	\text{s.c.}  & \ma^\dagger(\G_{XY}(\vx_{i})) \in \setC, \forall i \nonumber \enspace .
\end{eqnarray}
% 	
With such a membership constraint, the projection operator $\Pi_{\setC}$ on $\setC$ can be defined as the solution to the optimization problem
%	
\begin{equation}
	\label{eq:projection_op}
	\min_{(\vr, \vu) \in \setC} \frac{1}{2} \| (\vs_0, \vs_{1,2}) - (\vr, \vu)\|_2^2 \enspace,
\end{equation}
%
which has a closed-form  \citep{Parikh2014} as
%	
\begin{equation}
	\label{eq:proj_lorentz_cone_2}
	\Pi_{\setC}(\vs_0, \vs_{1,2}) = \left \lbrace
	\begin{array}{lll}
		(\vs_0, \vs_{1,2}) & \text{if} & \| \vs_{1,2} \|_2 \leq \vs_0 \\
		\frac{1+ \vs_0/\|\vs_{1,2}\|_2}{2} (\|\vs_{1,2}\|_2, \vs_{1,2}) & \text{if} & \| \vs_{1,2} \|_2 > \vs_0
	\end{array}
	\right.
\end{equation}
%
We can introduce this projection operator to the training algorithm instead of the loss term $L_{\setC_2}$ obtain by the relaxation of $\setC_2$.  To do so, the output of $\G_{XY}$ is systematically projected onto $\setC$ using $\Pi_{\setC}$ as
%
\begin{equation}
	\Hat{\vy}_{\Pi_{\setC}} = \ma\Pi_{\setC}(\ma^\dagger\G_{XY(\vx)}) \enspace,
\end{equation} 
%
with $\vx \in \spaceX$ an \ac{RGB} image. This process is summed up in Algorithm \ref{alg:cyclegan_train_projection} and illustrated in Figure \ref{fig:overview_polarCycle_proj}.

\begin{algorithm}[!t]
	\begin{algorithmic}[]
		\REQUIRE{$\trainsetX$ and $\trainsetY$ two unpaired datasets, $\Gxy$ and $\Gyx$ the mapping networks, $\Dx$ and $\Dy$ the discrimination models, $m$ the mini-batch size, $\ma$ the calibration matrix and $\ma^\dagger$ its pseudo-inverse, $\lambda, \mu, \nu$ hyperparameters}
		\REPEAT
		\STATE sample a mini-batch $\lbrace \vx_i \rbrace_{i=1}^m$ from $\trainsetX$\;
		\STATE sample a mini-batch $\lbrace \vy_i \rbrace_{i=1}^m$ from $\trainsetY$\;
		\STATE update $\Dx$ by stochastic gradient descent of
		\STATE \ \ \ \ $ \sum_{i=1}^{m}\big(\Dx(\vx_i)-1)^2 + (\Dx(\Gyx(\vy_i))\big)^2$
		\STATE update $\Dy$ by stochastic gradient descent of
		\STATE \ \ \ \ $ \sum_{i=1}^{m}\big(\Dy(\vy_i)-1)^2 + (\Dy(\Gxy(\vx_i))\big)^2$
		\STATE \textbf{for} $i=1$ to $n$, compute $\hat{\vs_i} = \begin{bmatrix}	\Hat{\vs_{0_i}} & 	\Hat{\vs_{1_i}} & 	\Hat{\vs_{2_i}}\end{bmatrix}^\top = \ma^\dagger\G_{XY}(\vx_i)$.
		\STATE update $\Gxy$ by stochastic gradient descent of
		\STATE \ \ \ \ $ \sum_{i=1}^n\big(\Dy(\Gxy(\vx_i))-1\big)^2 + \lambda \big(||\vx_i - \Gyx(\Gxy(\vx_i)\big)||_1 +||\vy_i -\Gxy(\Gyx(\vy_i))||_1\big)$
		\STATE \ \ \ \ \ \ $+ \mu \big(\|\vx_i - \ma\ma^\dagger\G_{XY}(\vx_i)\|^2_F\Big)  + \nu \|\ma^\dagger \G_{XY}(\vx_i) - \Pi_\setC(\ma^\dagger \G_{XY}(\vx_i))\|^2$\;
		\STATE update $\Gyx$ by stochastic gradient descent of
		\STATE \ \ \ \ $ \sum_{i=1}^n \big(\Dx(\Gyx(\vy_i))-1\big)^2+ \lambda \big(||\vx_i - \Gyx(\Gxy(\vx_i))||_1 + ||\vy_i - \Gxy(\Gyx(\vy_i))||_1\big)$\;
		\UNTIL a stopping condition is met
	\end{algorithmic}
	\caption{CycleGAN with proximal training algorithm}
	\label{alg:cyclegan_train_proximal}
\end{algorithm}

\begin{figure} 
	\centering
	\includegraphics[width=\textwidth*2/3]{ours_prox}
	\caption{Overview of the CycleGAN training process extended with the $L_{\setC_1}$ and proximal losses}
	\label{fig:overview_polarCycle_prox}
\end{figure}

\subsection{Proximal method for generating polarimetric images}

Another solution is to formulate an alternative version to the loss induced by relaxation (\ref{eqn:lreg}) that measures the distance between the Stokes vectors of the generated image and their projection on the constraint space, as
%	
\begin{equation}
	\label{eq:polar_proximal_formulation}
	L_{prox} (\G_{XY}) = \mathop{\mathbb{E}}_{\vx\sim \p{X}} \Big[\|\ma^\dagger \G_{XY}(\vx) - \Pi_\setC(\ma^\dagger \G_{XY}(\vx))\|^2\Big] \enspace  ,
\end{equation}
%	
with $\lambda$ a regularization parameter. This loss can be substituted to $L_{\setC_2}$ in Equation \ref{eqn:lfinal}, thus the problem becomes
%
\begin{equation}
	L_{final}(\G_{XY}, \G_{YX}, \D_X, \D_Y)= L_{CycleGAN}(\G_{XY}, \G_{YX}, \D_X, \D_Y)+\mu L_{\mathcal{C}_1}(\G_{XY}) + \nu L_{prox}(\G_{XY}) \enspace.
	\label{eqn:lfinal_prox}
\end{equation}
%
This process is summed up in Algorithm \ref{alg:cyclegan_train_proximal} and illustrated in Figure \ref{fig:overview_polarCycle_prox}. Note that the gradient of the distance $\Omega_\setC = \|\ma^\dagger \G_{XY}(\vy) - \Pi_\setC(\ma^\dagger \G_{XY}(\vy))\|^2$ can be expressed \citep{Parikh2014} as 
%
\begin{equation}
	\nabla_{\G_{XY}} \Omega_{\setC}(\vs) = \left( \vs - \Pi_{\setC}(\vs) \right) \times \left \lbrace
	\begin{array}{lll}
		0 & \text{if} & \|\vs_{1,2}\|_2 \leq \vs_0 \\
		\nabla_{\G_{XY}} \vs - \nabla_{\G_{XY}}   \frac{1}{2} \big[(1+ \frac{\vs_0}{\|\vs_{1,2}\|_2}) (\|\vs_{1,2}\|_2, \vs_{1,2})\big]  & \text{if} & \|\vs_{1,2} \|_2 > \vs_0
	\end{array}
	\right.
	\label{eq:grad_lorentz_cone_3}
\end{equation}
%
Such an approach for learning models with constraints has been used, for example, by \citet{Kervadec2019} as an alternative to the Lagrangian version of an image segmentation problem under volume constraints determined by a convolutional neural network \citep{Pathak2015}.


\section{Conclusion}

In this work, we proposed an efficient way to generate realistic polarimetric images subject to physical admissibility constraints. An adapted \ac{CycleGAN} is used to achieve the generation of pixel-wise physical images. To train the proposed output-constrained CycleGAN, we combined the standard \ac{CycleGAN}'s objective function with two designed cost functions in order to handle the feasibility constraints related to each polarization-encoded pixel in the image. With the proposed generative model, we successfully translated RGB images from road scenes to polarimetric images showing an enhancement of the detection performances.

As a perspective, we characterize the set of the constraints and formulate a projection operator on this set. Using this operator, we propose two additional algorithms for transferring color images into the polarimetric domain: the first one consist in projecting the output of the generator to the space of the constraints and giving these projected images to the discriminator; and the second algorithm consist in formulating a proximal distance between the projected samples and the generated ones, and adding this distance as an auxiliary task to the CycleGAN's objective.

Another future work  direction would  be to improve the quality of the small objects in generated images in order to enhance the performances of the road-scene analysis models on, for example, pedestrian detection.

It would also be interesting to extend the generation of polarimetric images to other domains such as medical and Synthetic-Aperture Radar \citep{vanZyl2011} imaging. Extension of the generation procedure to road scene images under adverse weather conditions may help improving object detection in these situations.