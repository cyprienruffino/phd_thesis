\chapter{Conditioning generation with multiple task-specific constraints}
\label{chap:chapter3}

\begin{chapterabstract}
	content...
\end{chapterabstract}

\minitoc

\section{Introduction}
Formulation as a constrained optimization problem

Reformulation of CycleGAN as a constrained optimization problem 

Relaxation of the constraints

Ici, expérimenter sur des datasets artificiels ?

\section{Proximal method for non-Euclidean output space}
Travail sur le proximal ?

Envelope theorem application

\section{Application to RGB to Polarimetric domain transfer}

Introduction to polarimetry-specific physical constraints (briefly, no need to write a physics essay)

Reformulation as constraints on the output space

Relaxations : $L_2$ term + rectified term

Dataset, Evaluation

Experiments and results

\section{Conclusion}

Relaxation of the constraints works even when a lot of constraints are applied

The application to the polarimetric dataset works

Future works : using adapted metrics for the non-euclidean outspace