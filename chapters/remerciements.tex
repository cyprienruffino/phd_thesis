\chapter*{Remerciements}
%\addcontentsline{toc}{chapter}{Remerciements}
\label{chap:remerciements}

Il est difficile d'estimer l'impact que les autres peuvent avoir sur notre vie. Cependant, si il est un témoignage de cet impact sur la mienne, c'est l'existence de ce manuscrit. A ce titre, j'ai bien des personnes à remercier.

J'aimerais donc, pour commencer chronologiquement, remercier mes professeurs de calculabilité et complexité, qui ont contribué à mon éventuelle orientation vers la thèse. : Nicolas Ollinger, Matthieu Liedloff et Ioan Todinca; pour m'avoir donné envie de m'orienter vers la recherche, même si finalement mon choix s'est orienté, merci Achille, vers le \textit{machine learning}. J'aimerais également remercier Thierry Paquet, Yann Soullard et Christopher Kermorvant de m'avoir fait suffisamment confiance pour m'embaucher en tant que stagiaire au LITIS. Non seulement cela m'a permis de me former au \textit{machine learning}, mais cela m'aura également permis de rencontrer mes encadrants de thèse.

Je ne remercierai jamais assez Gilles Gasso et Romain Hérault pour leur supervision, pour toute la science que l'on a pu partager, pour toute la patience qu'ils ont sû m'accorder, et surtout pour leur bienveillance. Leur présence tout au long des 4 ans de cette thèse aura été indispensable à l'existence de cette thèse.

J'aimerai également remercier tous mes camarades du LITIS. Que ce soit côté fac': Guillaume, Achille, Rosana, Yann, Andres, Wassim, Gaëtan, Sen ou Fabrice; comme côté INSA: Ismaïla, Rachel,  Franco, Matthieu, Ben, JB, Flavie, Linlin, Nikolas ou Soufiane. Que ce soit pour toutes les discussions dans les couloirs ou pour les parties de tarot à la pause café, l'ambiance au labo est excellente et les discussions toujours enrichissantes. Je remercie également toutes mes collaborateurs, notamment Samia Ainouz, Stéphane Canu, Rachel Blin et Eric Laloy. Je remercie également mon jury de thèse de l'intérêt qu'ils ont porté à mes travaux et pour avoir accepté de relire ma thèse et participer à ma soutenance.

Ensuite, j'aimerai énormément remercier les copains Rouennais. Achille, Diego, Victor, Lauréline, Val', Cranky, et de manière générale toute la "team cafard". Votre accueil sans égal au pays de la crème fraiche et du cidre m'aura fait passer de sacrés bons moments et aura bien contribué à abaisser la pression du doctorat.  Merci également aux amis de longue date, Nico, Cassandre, Najib, Romu et Pierre, c'est toujours une joie incroyable de se retrouver dès que l'occasion s'y prête.

Vient enfin le tour de ma famille. Je pense que je ne serai jamais capable d'exprimer comme je le souhaiterai la reconnaissance que j'ai envers eux. De la famille plus éloignée à la plus proche, votre bienveillance et votre attention ont été le carburant de mon travail. Mais c'est évidemment à mes parents et mes frères-et-sœurs que je dois le plus. M'man, P'pa, Gab, Flo, Zach, je sais que vous lirez éventuellement ce message. Vous êtes le plus grand soutien de ma vie, et pas une ligne de cette thèse n'existerai sans vous. Dans les moments agréables comme dans les plus difficiles, vous avez toujours été là. Je ne vous en remercierai jamais assez.

- Cyprien