\chapter{Proof of the polarimetric manifold}
\label{app:physical_prop}

\begin{proof}
	We call $\hat{\vx}$ and $\tilde{\vx}$ respectively an image that we want to evaluate and the intensities computed by equation (\ref{eq:calibration_constraint}). From this equation, the calibration matrix $\ma$ and its pseudo-inverse $\ma^\dagger$, we have the following equality:
	\begin{equation}
			\label{eq:app_problem}
		\begin{bmatrix} 
			\tilde{\vx}_0 \\
			\tilde{\vx}_{45} \\	
			\tilde{\vx}_{90} \\
			\tilde{\vx}_{135} 
		\end{bmatrix} = \frac{1}{2}\begin{bmatrix}
			1 & 1 & 0 \\
			1 & 0 & 1 \\
			1 & -1 & 0 \\
			1 & 0 & -1
		\end{bmatrix}
		\times \begin{bmatrix}
			1 & 0 & 1 & 0 \\
			1 & 0 & -1 & 0 \\
			0 & 1 & 0 & -1
		\end{bmatrix} \times \begin{bmatrix} 
			\hat{\vx}_0 \\
			\hat{\vx}_{45} \\
			\hat{\vx}_{90} \\
			\hat{\vx}_{135}
		\end{bmatrix}\enspace.
	\end{equation}
	
	Let $\mm = \ma\ma^\dagger$, then we have:
	$$
	\tilde{\vx} = \mm\hat{\vx} = \frac{1}{2}\begin{bmatrix}
		2 & 0 & 0 & 0 \\
		1 & 1 & 1 & -1 \\
		0 & 0 & 2 & 0 \\
		1 & -1 & 1 & 1
	\end{bmatrix} \times \begin{bmatrix} 
		\hat{\vx}_0 \\
		\hat{\vx}_{45} \\
		\hat{\vx}_{90} \\
		\hat{\vx}_{135}
	\end{bmatrix} \enspace.
	$$
	
	The set $\setX$ such that its elements are solutions to Problem \ref{eq:app_problem} is
	$$
	\setX = \{\vx | \vx = \mm\vx\} = \{\vx | (\mm - \mi)\vx = 0\}  = Ker(\mm - \mi)\enspace.
	$$
	
	Lets first compute the matrix $\mm-\mi$:
	\begin{equation*}
		\begin{split}
			\mm - I = \frac{1}{2} \left( 
			\begin{bmatrix}
				2 & 0 & 0 & 0 \\
				1 & 1 & 1 & -1 \\
				0 & 0 & 2 & 0 \\
				1 & -1 & 1 & 1
			\end{bmatrix} -
			\begin{bmatrix}
				2 & 0 & 0 & 0 \\
				0 & 2 & 0 & 0 \\
				0 & 0 & 2 & 0 \\
				0 & 0 & 0 & 2
			\end{bmatrix}
			\right)
			= \frac{1}{2}\begin{bmatrix}
				0 & 0 & 0 & 0 \\
				1 & -1 & 1 & -1 \\
				0 & 0 & 0 & 0 \\
				1 & -1 & 1 & -1
			\end{bmatrix}
		\end{split} \enspace,
	\end{equation*}
	with $\mi$ the identity matrix. Lets now find $\vx$ such that $(\mm - \mi)\vx = 0$:
	$$
	\frac{1}{2}\begin{bmatrix}
		0 & 0 & 0 & 0 \\
		1 & -1 & 1 & -1 \\
		0 & 0 & 0 & 0 \\
		1 & -1 & 1 & -1
	\end{bmatrix} 
	\times
	\begin{bmatrix}
		\vx_1 \\
		\vx_2 \\
		\vx_3 \\
		\vx_4
	\end{bmatrix} = 
	\begin{bmatrix}
		0 \\
		0 \\
		0 \\
		0
	\end{bmatrix}
	$$
	Thus we have $\vx_1 - \vx_2 + \vx_3 - \vx_4 = 0$. Hence $\setX$ comprises vectors $\vx\in\spaceR^4$ with the constraint $\vx_1 + \vx_3 = \vx_2 + \vx_4$, leading to $\setX = \Big\{\begin{bmatrix}	\vx_0 & \vx_1 & \vx_2 & \vx_3 \end{bmatrix}^\top \Big|\enspace \vx_1 + \vx_3 = \vx_2 + \vx_4\Big\}$.
\end{proof}