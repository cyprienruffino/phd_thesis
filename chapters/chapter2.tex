\chapter{Reconstruction as an Auxiliary Task for Generative Modeling}
\label{chap:chapter2}
 \section{Image Reconstruction with Generative Models}

Introduction to the problem of image reconstruction, definitions and formulation

Related works : CGAN, AmbientGAN, UNIR,  Compressed Sensing with Meta-Learning

Limitations of these models

\section{Conditional generation as a Maximum A Posteriori estimation}
Approche de l'article NeuCom :

Formulation as a Maximum A Posteriori Estimation, assumptions (normal error)

Construction of the loss term using bayes rule and least squares 

PacGAN for keeping the diversity

\section{Experimental evaluation and application to underground soil generation}

Datasets : MNIST/FashionMNIST/CelebA/Texture

Evaluation : MSE/FID; Epoch  selection criterion

Architectures : Appendix ? DCGAN + SGAN (encoder-decoder)

Results : visible trade-off, good fidelity overall

Application to hydro-geology : subsurface dataset

Evaluation : MSE/HOG+LBP

\section{Conclusion}

Objective reached : tuneable loss, pixel-wise, keeping diversity

Applications in hydro-geology : papier Eric

Future works : other distributions (modelling error using Laplacian, beta or Poisson distributions)
