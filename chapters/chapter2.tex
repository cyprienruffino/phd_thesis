\chapter{Reconstruction as an Auxiliary Task for Generative Modeling}
\label{chap:chapter2}

\begin{chapterabstract}
	In this chapter, we propose an approach for conditioning a GAN model to reconstuct images from a very sparse set of randomly-positioned pixels known beforehand. This approach, based on a Maximum A Posteriori estimation, takes the form of an explicit auxiliary reconstruction task which adds to the GAN objective as an additional loss term. Complemented with the PacGAN variant for training GANs, this approach enables the generation of diverse samples from a sparse pixel map. As opposed to the more classical Conditional GAN approach, this auxiliary task is interpretable and a hyperparameter allows to control the importance of the conditioning in the learning process. We evaluate our approach on the classical MNIST, FashionMNIST and CIFAR10 datasets, as well as a custom-made texture dataset. Finally,  we apply this approach to a task of geostatistical simulation.
\end{chapterabstract}

 \section{Image Reconstruction with Generative Models}

Introduction to the problem of image reconstruction, definitions and formulation

Related works : CGAN, AmbientGAN, UNIR,  Compressed Sensing with Meta-Learning

Limitations of these models

\section{Conditional generation as a Maximum A Posteriori estimation}
Approche de l'article NeuCom :

Formulation as a Maximum A Posteriori Estimation, assumptions (normal error)

Construction of the loss term using bayes rule and least squares 

PacGAN for keeping the diversity

\section{Experimental evaluation and application to underground soil generation}

Datasets : MNIST/FashionMNIST/CelebA/Texture

Evaluation : MSE/FID; Epoch  selection criterion

Architectures : Appendix ? DCGAN + SGAN (encoder-decoder)

Results : visible trade-off, good fidelity overall

Application to hydro-geology : subsurface dataset

Evaluation : MSE/HOG+LBP

\section{Conclusion}

Objective reached : tuneable loss, pixel-wise, keeping diversity

Applications in hydro-geology : papier Eric

Future works : other distributions (modelling error using Laplacian, beta or Poisson distributions)
