%%%%%%%%%%%%%%%%%%%%%%%%%%%%%%%%%%%%%%%%
%           Commandes perso            %
%%%%%%%%%%%%%%%%%%%%%%%%%%%%%%%%%%%%%%%%

%% Figures centrées, et en position 'here, top, bottom or page'
\newenvironment{figureth}{%
		\begin{figure}[htbp]
			\centering
	}{
		\end{figure}
		}
		
		
%% Tableaux centrés, et en position 'here, top, bottom or page'
\newenvironment{tableth}{%
		\begin{table}[htbp]
			\centering
			%\rowcolors{1}{coleurtableau}{coleurtableau}
	}{
		\end{table}
		}

%% Sous-figures centrées, en position 'top'		
\newenvironment{subfigureth}[1]{%
	\begin{subfigure}[t]{#1}
	\centering
}{
	\end{subfigure}
}

\newcommand{\citationChap}[2]{%
	\epigraph{\og \textit{#1} \fg{}}{#2}
}

%% On commence par une page impaire quand on change le style de numérotation de pages 
\let\oldpagenumbering\pagenumbering
\renewcommand{\pagenumbering}[1]{%
	\cleardoublepage
	\oldpagenumbering{#1}
}

\newcommand{\GG}[1]{{\color{red} GG:  #1}}
\newcommand{\CR}[1]{{\color{blue} CR:  #1}}
\newcommand{\RH}[1]{{\color{green} RB:  #1}}

\newenvironment{chapterabstract}{{\center \Large \textbf{\textit{Chapter abstract}}\vspace{0.3cm}\\}\rightskip1in\itshape}{}

\newcommand{\cmark}{\ding{51}}%
\newcommand{\xmark}{\ding{55}}%

\newcommand{\setS}{\mathcal{S}}
\newcommand{\setX}{\mathcal{X}}
\newcommand{\setY}{\mathcal{Y}}
\newcommand{\setZ}{\mathcal{Z}}
\newcommand{\trainsetX}{\mathcal{D}_X}
\newcommand{\trainsetY}{\mathcal{D}_Y}
\newcommand{\vect}{\text{vect}}

\makeatletter
\newcommand{\acronym}[2]{
	\label{acronyms:#1}
	\@namedef{#1}{\hyperref[acronyms:#1]{#1}}%
	#1 & #2 \\
}

\newcommand{\acronyms}[3]{
	\label{acronyms:#1} 
	\@namedef{#1}{\hyperref[acronyms:#1]{#1}}%
	\@namedef{#2}{\hyperref[acronyms:#1]{#2}}%
	#1 & #3\\
}
\makeatother

