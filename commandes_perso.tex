%%%%%%%%%%%%%%%%%%%%%%%%%%%%%%%%%%%%%%%%
%           Commandes perso            %
%%%%%%%%%%%%%%%%%%%%%%%%%%%%%%%%%%%%%%%%

%% Figures centrées, et en position 'here, top, bottom or page'
\newenvironment{figureth}{%
		\begin{figure}[htbp]
			\centering
	}{
		\end{figure}
		}
		
		
%% Tableaux centrés, et en position 'here, top, bottom or page'
\newenvironment{tableth}{%
		\begin{table}[htbp]
			\centering
			%\rowcolors{1}{coleurtableau}{coleurtableau}
	}{
		\end{table}
		}

%% Sous-figures centrées, en position 'top'		
\newenvironment{subfigureth}[1]{%
	\begin{subfigure}[t]{#1}
	\centering
}{
	\end{subfigure}
}

%% On commence par une page impaire quand on change le style de numérotation de pages 
\let\oldpagenumbering\pagenumbering
\renewcommand{\pagenumbering}[1]{%
	\cleardoublepage
	\oldpagenumbering{#1}
}

\newcommand{\GG}[1]{{\color{red} GG:  #1}}
\newcommand{\CR}[1]{{\color{blue} CR:  #1}}
\newcommand{\RH}[1]{{\color{green} RB:  #1}}

\newenvironment{chapterabstract}{{\center \Large \textbf{\textit{Chapter abstract}}\vspace{0.3cm}\\}\rightskip1in\itshape}{}

\newcommand{\cmark}{\ding{51}}%
\newcommand{\xmark}{\ding{55}}%

% Pour les acronymes
\newcommand{\ac}[1]{\hyperref[acronyms:#1]{#1}}  % Pour référencer un acronyme
\newcommand{\acmake}[2]{\label{acronyms:#1}#1 & #2 \\}  % Pour créer un acronyme
\newcommand{\acdup}[1]{\label{acronyms:#1}} % Pour créer un alias à un acronyme 

% Pour le glossaire
\newcommand{\gl}[2]{\hyperref[gl:#1]{{\color{\cole}#2\footnote{see Glossary, Appendix \ref{gl:#1}}\setcounter{footnote}{0}}}}  % Pour référencer une entrée de glossaire

\newcommand{\glnf}[2]{\hyperref[gl:#1]{{\color{\cole}#2\footnotemark}}\setcounter{footnote}{0}}  % Pour référencer une entrée de glossaire


\newcommand{\GAN}{\ac{GAN}}
\newcommand{\GANs}{\ac{GAN}s}


\newcommand{\refchap}[2]{%
	\epigraph{\og \textit{#1} \fg{}}{#2}
}

\newcommand{\citeq}[1]{Equation \eqref{#1}}
\newcommand{\citealg}[1]{Algorithm \ref{#1}}
\newcommand{\citesec}[1]{Section \ref{#1}}
\newcommand{\citesub}[1]{Subsection \ref{#1}}
\newcommand{\citefig}[1]{Figure \ref{#1}}
\newcommand{\seefigure}[1]{see \citefig{#1}}

\newcommand{\bigrule}{\specialrule{.2em}{.1em}{.1em}}
\newcommand{\Bigrule}{\specialrule{.3em}{.2em}{.2em}}